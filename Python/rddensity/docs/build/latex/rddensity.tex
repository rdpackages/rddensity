%% Generated by Sphinx.
\def\sphinxdocclass{report}
\documentclass[letterpaper,10pt,english]{sphinxmanual}
\ifdefined\pdfpxdimen
   \let\sphinxpxdimen\pdfpxdimen\else\newdimen\sphinxpxdimen
\fi \sphinxpxdimen=.75bp\relax
\ifdefined\pdfimageresolution
    \pdfimageresolution= \numexpr \dimexpr1in\relax/\sphinxpxdimen\relax
\fi
%% let collapsible pdf bookmarks panel have high depth per default
\PassOptionsToPackage{bookmarksdepth=5}{hyperref}

\PassOptionsToPackage{warn}{textcomp}
\usepackage[utf8]{inputenc}
\ifdefined\DeclareUnicodeCharacter
% support both utf8 and utf8x syntaxes
  \ifdefined\DeclareUnicodeCharacterAsOptional
    \def\sphinxDUC#1{\DeclareUnicodeCharacter{"#1}}
  \else
    \let\sphinxDUC\DeclareUnicodeCharacter
  \fi
  \sphinxDUC{00A0}{\nobreakspace}
  \sphinxDUC{2500}{\sphinxunichar{2500}}
  \sphinxDUC{2502}{\sphinxunichar{2502}}
  \sphinxDUC{2514}{\sphinxunichar{2514}}
  \sphinxDUC{251C}{\sphinxunichar{251C}}
  \sphinxDUC{2572}{\textbackslash}
\fi
\usepackage{cmap}
\usepackage[T1]{fontenc}
\usepackage{amsmath,amssymb,amstext}
\usepackage{babel}



\usepackage{tgtermes}
\usepackage{tgheros}
\renewcommand{\ttdefault}{txtt}



\usepackage[Bjarne]{fncychap}
\usepackage{sphinx}

\fvset{fontsize=auto}
\usepackage{geometry}


% Include hyperref last.
\usepackage{hyperref}
% Fix anchor placement for figures with captions.
\usepackage{hypcap}% it must be loaded after hyperref.
% Set up styles of URL: it should be placed after hyperref.
\urlstyle{same}

\addto\captionsenglish{\renewcommand{\contentsname}{Contents:}}

\usepackage{sphinxmessages}
\setcounter{tocdepth}{1}



\title{rddensity}
\date{May 27, 2022}
\release{2.2.0}
\author{Matias Cattaneo, Rajita Chandak, Michael Jansson, Xinwei Ma}
\newcommand{\sphinxlogo}{\vbox{}}
\renewcommand{\releasename}{Release}
\makeindex
\begin{document}

\pagestyle{empty}
\sphinxmaketitle
\pagestyle{plain}
\sphinxtableofcontents
\pagestyle{normal}
\phantomsection\label{\detokenize{index::doc}}


\sphinxAtStartPar
Density discontinuity testing (a.k.a. manipulation testing)
is commonly employed in regression discontinuity designs
and other program evaluation settings to detect perfect self\sphinxhyphen{}selection (manipulation)
around a cutoff where treatment/policy assignment changes.
This package implements manipulation testing procedures using
the local polynomial density estimators:
\sphinxcode{\sphinxupquote{rddensity()}} to construct test statistics and p\sphinxhyphen{}values given a prespecified cutoff,
\sphinxcode{\sphinxupquote{rdbwdensity()}} to perform data\sphinxhyphen{}driven bandwidth selection,
and \sphinxcode{\sphinxupquote{rdplotdensity()}} to construct density plots.

\sphinxAtStartPar
Additional information regarding rddensity is available \sphinxhref{https://rdpackages.github.io/rddensity\_doc/}{here}.

\sphinxAtStartPar
Companion \sphinxcode{\sphinxupquote{Stata}} aand \sphinxcode{\sphinxupquote{R}} packages
and additional regression discontinuity packages
are available at \sphinxhref{https://rdpackages.github.io/}{rdpackages.github.io}.

\sphinxAtStartPar
Install \sphinxcode{\sphinxupquote{rddensity}} by running
\sphinxcode{\sphinxupquote{pip install rddensity}}.

\sphinxAtStartPar
Import functions by running

\begin{sphinxVerbatim}[commandchars=\\\{\}]
\PYG{g+gp}{\PYGZgt{}\PYGZgt{}\PYGZgt{} }\PYG{k+kn}{from} \PYG{n+nn}{rddensity} \PYG{k+kn}{import} \PYG{n}{rddensity}
\end{sphinxVerbatim}

\begin{sphinxVerbatim}[commandchars=\\\{\}]
\PYG{g+gp}{\PYGZgt{}\PYGZgt{}\PYGZgt{} }\PYG{k+kn}{from} \PYG{n+nn}{rddensity} \PYG{k+kn}{import} \PYG{n}{rdbwdensity}
\end{sphinxVerbatim}

\begin{sphinxVerbatim}[commandchars=\\\{\}]
\PYG{g+gp}{\PYGZgt{}\PYGZgt{}\PYGZgt{} }\PYG{k+kn}{from} \PYG{n+nn}{rddensity} \PYG{k+kn}{import} \PYG{n}{rdplotdensity}
\end{sphinxVerbatim}

\sphinxAtStartPar
Source code and replication files are available in the \sphinxhref{https://github.com/rdpackages/rddensity/}{rddensity repository}.


\chapter{References}
\label{\detokenize{index:references}}
\sphinxAtStartPar
Cattaneo M. D., M. Jansson, and X. Ma. 2018.
\sphinxhref{https://rdpackages.github.io/references/Cattaneo-Jansson-Ma\_2018\_Stata.pdf}{Manipulation Testing based on Density Discontinuity.}.
\sphinxstyleemphasis{Stata Journal} 18(1): 234\sphinxhyphen{}261.

\sphinxAtStartPar
Cattaneo M. D., M. Jansson, and X. Ma. 2022.
\sphinxhref{https://rdpackages.github.io/references/Cattaneo-Jansson-Ma\_2022\_JSS.pdf}{lpdensity: Local Polynomial Density Estimation and Inference.}.
\sphinxstyleemphasis{Journal of Statistical Software} Forthcoming.


\chapter{Authors}
\label{\detokenize{index:authors}}
\sphinxAtStartPar
Matias D. Cattaneo, Princeton University. (\sphinxhref{mailto:cattaneo@princeton.edu}{cattaneo@princeton.edu}).

\sphinxAtStartPar
Rajita Chandak (maintainer), Princeton University. (\sphinxhref{mailto:rchandak@princeton.edu}{rchandak@princeton.edu}).

\sphinxAtStartPar
Michael Jansson, University of California Berkeley. (\sphinxhref{mailto:mjansson@econ.berkeley.edu}{mjansson@econ.berkeley.edu}).

\sphinxAtStartPar
Xinwei Ma (maintainer), University of California San Diego. (\sphinxhref{mailto:x1ma@ucsd.edu}{x1ma@ucsd.edu}).

\sphinxstepscope


\section{rddensity}
\label{\detokenize{modules:rddensity}}\label{\detokenize{modules::doc}}
\sphinxstepscope


\subsection{rddensity}
\label{\detokenize{rddensity:rddensity}}\label{\detokenize{rddensity::doc}}
\sphinxAtStartPar
Manipulation testing using local polynomial density estimation


\subsubsection{Description}
\label{\detokenize{rddensity:description}}
\sphinxAtStartPar
\sphinxcode{\sphinxupquote{rddensity}} implements manipulation testing procedures using the
local polynomial density estimators proposed in Cattaneo, Jansson and Ma (2020),
and implements graphical procedures with valid confidence bands using the results in
Cattaneo, Jansson and Ma (2021a,b).
In addition, the command provides complementary manipulation testing based on finite
sample exact binomial testing following the results in
Cattaneo, Frandsen and Titiunik (2015) and Cattaneo, Frandsen and Vazquez\sphinxhyphen{}Bare (2017).
For an introduction to manipulation testing see McCrary (2008).

\sphinxAtStartPar
Companion commands: \sphinxcode{\sphinxupquote{rdbwdensity}} for bandwidth selection and \sphinxcode{\sphinxupquote{rdplotdensity}} for plotting estimation results.


\subsubsection{References}
\label{\detokenize{rddensity:references}}
\sphinxAtStartPar
Cattaneo M. D., B. Grandsen, and R. Titiunik. 2015
\sphinxhref{https://rdpackages.github.io/references/Cattaneo-Frandsen-Titiunik\_2015\_JCI.pdf}{Randomization Inference in the Regression Discontinuity Design: An Application to the Study of Party Advantages in the U.S. Senate.}
\sphinxstyleemphasis{Journal of Causal Inference} 3(1): 1\sphinxhyphen{}24.

\sphinxAtStartPar
Cattaneo M. D., M. Jansson, and X. Ma. 2018.
\sphinxhref{https://rdpackages.github.io/references/Cattaneo-Jansson-Ma\_2018\_Stata.pdf}{Manipulation Testing based on Density Discontinuity.}.
\sphinxstyleemphasis{Stata Journal} 18(1): 234\sphinxhyphen{}261.

\sphinxAtStartPar
Cattaneo, M. D., M. Jansson, and X. Ma. 2020.
\sphinxhref{https://nppackages.github.io/references/Cattaneo-Jansson-Ma\_2020\_JASA.pdf}{Simple Local Polynomial Density Estimators}.
\sphinxstyleemphasis{Journal of the American Statistical Association}, 115(531): 1449\sphinxhyphen{}1455.

\sphinxAtStartPar
Cattaneo M. D., M. Jansson, and X. Ma. 2022.
\sphinxhref{https://rdpackages.github.io/references/Cattaneo-Jansson-Ma\_2022\_JSS.pdf}{lpdensity: Local Polynomial Density Estimation and Inference.}.
\sphinxstyleemphasis{Journal of Statistical Software} Forthcoming.

\sphinxAtStartPar
Cattaneo M. D., R. Titiunik, and F. Vazquez\sphinxhyphen{}Bare. 2017.
\sphinxhref{https://rdpackages.github.io/references/Cattaneo-Titiunik-VazquezBare\_2017\_JPAM.pdf}{Comparing Inference Approaches for RD Designs: A Reexamination of the Effect of Head Start of Child Mortality}
\sphinxstyleemphasis{Journal of Policy Analysis and Management}, 36(3): 643:681.

\sphinxAtStartPar
McCrary, J. 2008.
Manipulation of the Running Variable in the Regression Discontinuity Design: A Density Test.
\sphinxstyleemphasis{Journal of Econometrics} 142(2): 698\sphinxhyphen{}714.


\subsubsection{Authors}
\label{\detokenize{rddensity:authors}}
\sphinxAtStartPar
Matias D. Cattaneo, Princeton University. (\sphinxhref{mailto:cattaneo@princeton.edu}{cattaneo@princeton.edu}).

\sphinxAtStartPar
Rajita Chandak (maintainer), Princeton University. (\sphinxhref{mailto:rchandak@princeton.edu}{rchandak@princeton.edu}).

\sphinxAtStartPar
Michael Jansson, University of California Berkeley. (\sphinxhref{mailto:mjansson@econ.berkeley.edu}{mjansson@econ.berkeley.edu}).

\sphinxAtStartPar
Xinwei Ma (maintainer), University of California San Diego. (\sphinxhref{mailto:x1ma@ucsd.edu}{x1ma@ucsd.edu}).

\phantomsection\label{\detokenize{rddensity:module-rddensity.rddensity}}\index{module@\spxentry{module}!rddensity.rddensity@\spxentry{rddensity.rddensity}}\index{rddensity.rddensity@\spxentry{rddensity.rddensity}!module@\spxentry{module}}\index{CJMrddensity (class in rddensity.rddensity)@\spxentry{CJMrddensity}\spxextra{class in rddensity.rddensity}}

\begin{fulllineitems}
\phantomsection\label{\detokenize{rddensity:rddensity.rddensity.CJMrddensity}}
\pysigstartsignatures
\pysiglinewithargsret{\sphinxbfcode{\sphinxupquote{class\DUrole{w}{  }}}\sphinxcode{\sphinxupquote{rddensity.rddensity.}}\sphinxbfcode{\sphinxupquote{CJMrddensity}}}{\emph{\DUrole{n}{hat}}, \emph{\DUrole{n}{sd\_asy}}, \emph{\DUrole{n}{sd\_jk}}, \emph{\DUrole{n}{test}}, \emph{\DUrole{n}{hat\_p}}, \emph{\DUrole{n}{sd\_asy\_p}}, \emph{\DUrole{n}{sd\_jk\_p}}, \emph{\DUrole{n}{test\_p}}, \emph{\DUrole{n}{n}}, \emph{\DUrole{n}{h}}, \emph{\DUrole{n}{fitselect}}, \emph{\DUrole{n}{kernel}}, \emph{\DUrole{n}{vce}}, \emph{\DUrole{n}{c}}, \emph{\DUrole{n}{p}}, \emph{\DUrole{n}{q}}, \emph{\DUrole{n}{regularize}}, \emph{\DUrole{n}{nLocalMin}}, \emph{\DUrole{n}{bino\_flag}}, \emph{\DUrole{n}{nUniqueMin}}, \emph{\DUrole{n}{massPoints}}, \emph{\DUrole{n}{massPoints\_flag}}, \emph{\DUrole{n}{bwselectl}}, \emph{\DUrole{n}{bwselect}}, \emph{\DUrole{n}{binoN}}, \emph{\DUrole{n}{binoW}}, \emph{\DUrole{n}{binoNStep}}, \emph{\DUrole{n}{binoWStep}}, \emph{\DUrole{n}{binoNW}}, \emph{\DUrole{n}{binoP}}, \emph{\DUrole{n}{useall}}, \emph{\DUrole{n}{X\_min}}, \emph{\DUrole{n}{X\_max}}, \emph{\DUrole{n}{bino}}}{}
\pysigstopsignatures
\sphinxAtStartPar
Class of rddensity function outputs.
Object type returned by {\hyperref[\detokenize{rddensity:rddensity.rddensity.rddensity}]{\sphinxcrossref{\sphinxcode{\sphinxupquote{rddensity()}}}}}.

\end{fulllineitems}

\index{rddensity() (in module rddensity.rddensity)@\spxentry{rddensity()}\spxextra{in module rddensity.rddensity}}

\begin{fulllineitems}
\phantomsection\label{\detokenize{rddensity:rddensity.rddensity.rddensity}}
\pysigstartsignatures
\pysiglinewithargsret{\sphinxcode{\sphinxupquote{rddensity.rddensity.}}\sphinxbfcode{\sphinxupquote{rddensity}}}{\emph{\DUrole{n}{X}}, \emph{\DUrole{n}{c}\DUrole{o}{=}\DUrole{default_value}{0}}, \emph{\DUrole{n}{p}\DUrole{o}{=}\DUrole{default_value}{2}}, \emph{\DUrole{n}{q}\DUrole{o}{=}\DUrole{default_value}{0}}, \emph{\DUrole{n}{fitselect}\DUrole{o}{=}\DUrole{default_value}{\textquotesingle{}unrestricted\textquotesingle{}}}, \emph{\DUrole{n}{kernel}\DUrole{o}{=}\DUrole{default_value}{\textquotesingle{}triangular\textquotesingle{}}}, \emph{\DUrole{n}{vce}\DUrole{o}{=}\DUrole{default_value}{\textquotesingle{}jackknife\textquotesingle{}}}, \emph{\DUrole{n}{h}\DUrole{o}{=}\DUrole{default_value}{{[}{]}}}, \emph{\DUrole{n}{bwselect}\DUrole{o}{=}\DUrole{default_value}{\textquotesingle{}comb\textquotesingle{}}}, \emph{\DUrole{n}{useall}\DUrole{o}{=}\DUrole{default_value}{False}}, \emph{\DUrole{n}{massPoints}\DUrole{o}{=}\DUrole{default_value}{True}}, \emph{\DUrole{n}{regularize}\DUrole{o}{=}\DUrole{default_value}{True}}, \emph{\DUrole{n}{nLocalMin}\DUrole{o}{=}\DUrole{default_value}{None}}, \emph{\DUrole{n}{nUniqueMin}\DUrole{o}{=}\DUrole{default_value}{None}}, \emph{\DUrole{n}{bino\_flag}\DUrole{o}{=}\DUrole{default_value}{True}}, \emph{\DUrole{n}{binoW}\DUrole{o}{=}\DUrole{default_value}{None}}, \emph{\DUrole{n}{binoN}\DUrole{o}{=}\DUrole{default_value}{None}}, \emph{\DUrole{n}{binoWStep}\DUrole{o}{=}\DUrole{default_value}{None}}, \emph{\DUrole{n}{binoNStep}\DUrole{o}{=}\DUrole{default_value}{None}}, \emph{\DUrole{n}{binoNW}\DUrole{o}{=}\DUrole{default_value}{{[}10{]}}}, \emph{\DUrole{n}{binoP}\DUrole{o}{=}\DUrole{default_value}{{[}0.5{]}}}}{}
\pysigstopsignatures\begin{quote}\begin{description}
\item[{Parameters}] \leavevmode\begin{description}
\item[{\sphinxstylestrong{X: Numeric vector or one dimentional matrix/dataframe}}] \leavevmode
\sphinxAtStartPar
the running variable.

\item[{\sphinxstylestrong{c: Numeric}}] \leavevmode
\sphinxAtStartPar
Specifies the threshold or cutoff value in the support of \sphinxstyleemphasis{X}. Default is \sphinxstyleemphasis{0}.

\item[{\sphinxstylestrong{p: Nonnegative integer}}] \leavevmode
\sphinxAtStartPar
specifies the local polynomial order used to construct the density estimators. Default is \sphinxstyleemphasis{2} (local quadratic approximation).

\item[{\sphinxstylestrong{fitselect: String}}] \leavevmode
\sphinxAtStartPar
specifies the density estimation method. \sphinxstyleemphasis{unrestricted} (Default) for density estimation without any restrictions (two\sphinxhyphen{}sample, unrestricted inference). \sphinxstyleemphasis{restricted} for density estimation assuming equal distribution function and higher order dericatives.

\item[{\sphinxstylestrong{kernel: String}}] \leavevmode
\sphinxAtStartPar
specifies the kernel function used to construct the local polynomial estimators. Accepted kernels: \sphinxstyleemphasis{triangular} (Default), \sphinxstyleemphasis{epanechnikov} or \sphinxstyleemphasis{uniform}.

\item[{\sphinxstylestrong{vce: String}}] \leavevmode
\sphinxAtStartPar
specifies the procedure used to compute the variance\sphinxhyphen{}covariance matrix estimatior. \sphinxstyleemphasis{jackknife} (Default) for jackknife standard errors or \sphinxstyleemphasis{plugin} for asymptotic plug\sphinxhyphen{}in standard errors.

\item[{\sphinxstylestrong{massPoints: Boolean, Default *True*.}}] \leavevmode
\sphinxAtStartPar
Specifies wether to adjust for mass points in the data.

\item[{\sphinxstylestrong{useall: Boolean, Default *False*.}}] \leavevmode
\sphinxAtStartPar
If specified, will report two testing procedures: conventional test statistic (not valie when useing mse\sphinxhyphen{}optimal bandwidth) and robust bias\sphinxhyphen{}corrected statistic.

\item[{\sphinxstylestrong{h: Numeric}}] \leavevmode
\sphinxAtStartPar
Specifies the bandwidth used to construct the density estimators on the two sides of the cutoff. If not specified, the bandwidth h is computed using the companion function, \sphinxstyleemphasis{rdbwdensity}. If two bandwidths are specified, the first bandwidth is used for the data below the cutoff and the second bandwidth is used for the data above the cutoff.

\item[{\sphinxstylestrong{bwselect: String.}}] \leavevmode
\sphinxAtStartPar
Specified the bandwidth selection procedure to be used. \sphinxstyleemphasis{each}\sphinxhyphen{}based on MSE of each density estimator separately (two distinct bandwidths), \sphinxstyleemphasis{diff}\sphinxhyphen{} based on MSE of difference of two density estimators, gives one common bandwidth, \sphinxstyleemphasis{sum}\sphinxhyphen{}based on MSE of sum of two density estimators, gives one common bandwidth. \sphinxstyleemphasis{comb} (default)\sphinxhyphen{}bandiwdth is selected as a combination of the alternatives above. For \sphinxstyleemphasis{fitselect=’unrestricted’}, it selects \sphinxstyleemphasis{median(each, diff, sum)}. For \sphinxstyleemphasis{fitselect=’restricted’}, it selects \sphinxstyleemphasis{min(diff, sum)}

\item[{\sphinxstylestrong{regularize: Boolean, Default *True*.}}] \leavevmode
\sphinxAtStartPar
Specifies whether to conduct local sample size checking. When True, the bandwidth is chosen such that the local region includes at least \sphinxstyleemphasis{nLocalMin} observations and at least \sphinxstyleemphasis{nUniqueMin} unique observations.

\item[{\sphinxstylestrong{nLocalMin: Nonnegative integer}}] \leavevmode
\sphinxAtStartPar
Specifies the minimum number of observations in each local neighbourhood. This option will be ignored if set to \sphinxstyleemphasis{0} or if \sphinxstyleemphasis{regularize=False}. Default is \sphinxstyleemphasis{20+p+1}.

\item[{\sphinxstylestrong{nUniqueMin: Nonnegative integer}}] \leavevmode
\sphinxAtStartPar
Specifies the minimum number of unqieu observations in each local neighbourhood. This option will be ignored if set to \sphinxstyleemphasis{0} or if \sphinxstyleemphasis{regularize=False}. Default is \sphinxstyleemphasis{20+p+1}.

\item[{\sphinxstylestrong{bino: Boolean (Default True).}}] \leavevmode
\sphinxAtStartPar
Specifies whether to conduct binomial tests. By default the initial (smallest) window contains 20 observations, and its length is also used as the increment for subsequent windows.

\item[{\sphinxstylestrong{binoW: Numeric.}}] \leavevmode
\sphinxAtStartPar
Specifies the half length(s) of the initial window. If two values are provided, they will be used for the data below and above the cutoff separately.

\item[{\sphinxstylestrong{binoN: Nonnegative integer.}}] \leavevmode
\sphinxAtStartPar
Specifies the number of observations (closest to the cutoff) used for the binomial test. This is ignored if \sphinxstyleemphasis{binoW} is provided.

\item[{\sphinxstylestrong{binoWStep: Numeric.}}] \leavevmode
\sphinxAtStartPar
Specifies the increment in half lengths.

\item[{\sphinxstylestrong{binoNStep: Nonnegative integer.}}] \leavevmode
\sphinxAtStartPar
Specifies the increment in sample size. This is ignored if \sphinxstyleemphasis{binoWStep} is provided.

\item[{\sphinxstylestrong{binoNW: Nonnegative integer.}}] \leavevmode
\sphinxAtStartPar
Specifies the total number of windows. Default is \sphinxstyleemphasis{10}.

\item[{\sphinxstylestrong{binoP: Numeric.}}] \leavevmode
\sphinxAtStartPar
Specifies the null hypothesis of the binomial test. Default is \sphinxstyleemphasis{0.5}.

\end{description}

\item[{Returns}] \leavevmode\begin{description}
\item[{hat:}] \leavevmode
\sphinxAtStartPar
left/right: density estimate to the left/right of the cutoff. diff: difference in estimated densities on the two sides of the cutoff.

\item[{sd\_asy:}] \leavevmode
\sphinxAtStartPar
left/right: standard error for the estimated density to the left/right of the cutoff, diff: standard error for difference in estimated densities on the two sides of the cutoff. (based on asymptotic method)

\item[{sd\_jk:}] \leavevmode
\sphinxAtStartPar
left/right: standard error for the estimated density to the left/right of the cutoff, diff: standard error for difference in estimated densities on the two sides of the cutoff. (based on jackknife method)

\item[{test:}] \leavevmode
\sphinxAtStartPar
t\_asy/t\_jk: t statistic for the density discontinuity test. p\_asy/p\_jk: p\sphinxhyphen{}value for the density discontinuity test.

\item[{hat\_p:}] \leavevmode
\sphinxAtStartPar
Same as hat, without bias correction.

\item[{bino:}] \leavevmode
\sphinxAtStartPar
Binomial test results.

\item[{h:}] \leavevmode
\sphinxAtStartPar
bandwidth used to the left/right of the cutoff.

\item[{n:}] \leavevmode
\sphinxAtStartPar
full: full sample size, left/right: sample size to the left/right of the cutoff.

\item[{X\_min:}] \leavevmode
\sphinxAtStartPar
Smallest observations to the left and right of the cutoff.

\item[{X\_max:}] \leavevmode
\sphinxAtStartPar
Largest observations to the left and right of the cutoff.

\item[{options:}] \leavevmode
\sphinxAtStartPar
other options passed to the function are also stored within the object.

\end{description}

\end{description}\end{quote}


\sphinxstrong{See also:}
\nopagebreak

\begin{description}
\item[{{\hyperref[\detokenize{rdbwdensity:module-rddensity.rdbwdensity}]{\sphinxcrossref{\sphinxcode{\sphinxupquote{rddensity.rdbwdensity}}}}}}] \leavevmode
\item[{{\hyperref[\detokenize{rdplotdensity:module-rddensity.rdplotdensity}]{\sphinxcrossref{\sphinxcode{\sphinxupquote{rddensity.rdplotdensity}}}}}}] \leavevmode
\end{description}



\end{fulllineitems}



\subsubsection{Example}
\label{\detokenize{rddensity:example}}
\begin{sphinxVerbatim}[commandchars=\\\{\}]
\PYG{g+gp}{\PYGZgt{}\PYGZgt{}\PYGZgt{} }\PYG{k+kn}{import} \PYG{n+nn}{numpy} \PYG{k}{as} \PYG{n+nn}{np}
\PYG{g+gp}{\PYGZgt{}\PYGZgt{}\PYGZgt{} }\PYG{k+kn}{from} \PYG{n+nn}{rddensity} \PYG{k+kn}{import} \PYG{n}{rddensity}
\PYG{g+gp}{\PYGZgt{}\PYGZgt{}\PYGZgt{} }\PYG{n}{data} \PYG{o}{=} \PYG{n}{np}\PYG{o}{.}\PYG{n}{random}\PYG{o}{.}\PYG{n}{normal}\PYG{p}{(}\PYG{o}{\PYGZhy{}}\PYG{l+m+mf}{0.5}\PYG{p}{,}\PYG{l+m+mi}{1}\PYG{p}{,}\PYG{l+m+mi}{2000}\PYG{p}{)}
\PYG{g+gp}{\PYGZgt{}\PYGZgt{}\PYGZgt{} }\PYG{n}{rdd} \PYG{o}{=} \PYG{n}{rddensity}\PYG{p}{(}\PYG{n}{X}\PYG{o}{=}\PYG{n}{data}\PYG{p}{,} \PYG{n}{vce}\PYG{o}{=}\PYG{l+s+s2}{\PYGZdq{}}\PYG{l+s+s2}{jackknife}\PYG{l+s+s2}{\PYGZdq{}}\PYG{p}{)}
\PYG{g+gp}{\PYGZgt{}\PYGZgt{}\PYGZgt{} }\PYG{n+nb}{print}\PYG{p}{(}\PYG{n+nb}{repr}\PYG{p}{(}\PYG{n}{rdd}\PYG{p}{)}\PYG{p}{)}
\end{sphinxVerbatim}

\sphinxstepscope


\subsection{rdbwdensity}
\label{\detokenize{rdbwdensity:rdbwdensity}}\label{\detokenize{rdbwdensity::doc}}
\sphinxAtStartPar
Bandwidth selection for manipulation testing


\subsubsection{Description}
\label{\detokenize{rdbwdensity:description}}
\sphinxAtStartPar
\sphinxcode{\sphinxupquote{rdbwdensity}} implements several data\sphinxhyphen{}driven bandwidth selection methods useful to construct manipulation testing procedures using the local polynomial density estimators proposed in Cattaneo, Jansson and Ma (2020).

\sphinxAtStartPar
Related \sphinxcode{\sphinxupquote{Stata}} and \sphinxcode{\sphinxupquote{R}} useful for inference in regression discontinuity (RD) designs are available on the \sphinxhref{https://rdpackages.github.io}{rdpackages website}.

\sphinxAtStartPar
Companion commands: \sphinxcode{\sphinxupquote{rddensity}} for estimation and \sphinxcode{\sphinxupquote{rdplotdensity}} for plotting estimation results.


\subsubsection{References}
\label{\detokenize{rdbwdensity:references}}
\sphinxAtStartPar
Cattaneo M. D., M. Jansson, and X. Ma. 2018.
\sphinxhref{https://rdpackages.github.io/references/Cattaneo-Jansson-Ma\_2018\_Stata.pdf}{Manipulation Testing based on Density Discontinuity.}.
\sphinxstyleemphasis{Stata Journal} 18(1): 234\sphinxhyphen{}261.

\sphinxAtStartPar
Cattaneo M. D., M. Jansson, and X. Ma. 2020.
\sphinxhref{https://nppackages.github.io/references/Cattaneo-Jansson-Ma\_2020\_JASA.pdf}{Simple Local Polynomial Density Estimators}.
\sphinxstyleemphasis{Journal of the American Statistical Association}, 115(531): 1449\sphinxhyphen{}1455.


\subsubsection{Authors}
\label{\detokenize{rdbwdensity:authors}}
\sphinxAtStartPar
Matias D. Cattaneo, Princeton University. (\sphinxhref{mailto:cattaneo@princeton.edu}{cattaneo@princeton.edu}).

\sphinxAtStartPar
Rajita Chandak (maintainer), Princeton University. (\sphinxhref{mailto:rchandak@princeton.edu}{rchandak@princeton.edu}).

\sphinxAtStartPar
Michael Jansson, University of California Berkeley. (\sphinxhref{mailto:mjansson@econ.berkeley.edu}{mjansson@econ.berkeley.edu}).

\sphinxAtStartPar
Xinwei Ma (maintainer), University of California San Diego. (\sphinxhref{mailto:x1ma@ucsd.edu}{x1ma@ucsd.edu}).

\phantomsection\label{\detokenize{rdbwdensity:module-rddensity.rdbwdensity}}\index{module@\spxentry{module}!rddensity.rdbwdensity@\spxentry{rddensity.rdbwdensity}}\index{rddensity.rdbwdensity@\spxentry{rddensity.rdbwdensity}!module@\spxentry{module}}\index{bw\_output (class in rddensity.rdbwdensity)@\spxentry{bw\_output}\spxextra{class in rddensity.rdbwdensity}}

\begin{fulllineitems}
\phantomsection\label{\detokenize{rdbwdensity:rddensity.rdbwdensity.bw_output}}
\pysigstartsignatures
\pysiglinewithargsret{\sphinxbfcode{\sphinxupquote{class\DUrole{w}{  }}}\sphinxcode{\sphinxupquote{rddensity.rdbwdensity.}}\sphinxbfcode{\sphinxupquote{bw\_output}}}{\emph{\DUrole{n}{h}}, \emph{\DUrole{n}{n}}, \emph{\DUrole{n}{fitselect}}, \emph{\DUrole{n}{kernel}}, \emph{\DUrole{n}{vce}}, \emph{\DUrole{n}{c}}, \emph{\DUrole{n}{p}}, \emph{\DUrole{n}{regularize}}, \emph{\DUrole{n}{nLocalMin}}, \emph{\DUrole{n}{nUniqueMin}}, \emph{\DUrole{n}{massPoints}}, \emph{\DUrole{n}{massPoints\_flag}}, \emph{\DUrole{n}{X\_min}}, \emph{\DUrole{n}{X\_max}}}{}
\pysigstopsignatures
\sphinxAtStartPar
Class of rdbwdensity function outputs.
Object type returned by {\hyperref[\detokenize{rdbwdensity:rddensity.rdbwdensity.rdbwdensity}]{\sphinxcrossref{\sphinxcode{\sphinxupquote{rdbwdensity()}}}}}.

\end{fulllineitems}

\index{rdbwdensity() (in module rddensity.rdbwdensity)@\spxentry{rdbwdensity()}\spxextra{in module rddensity.rdbwdensity}}

\begin{fulllineitems}
\phantomsection\label{\detokenize{rdbwdensity:rddensity.rdbwdensity.rdbwdensity}}
\pysigstartsignatures
\pysiglinewithargsret{\sphinxcode{\sphinxupquote{rddensity.rdbwdensity.}}\sphinxbfcode{\sphinxupquote{rdbwdensity}}}{\emph{\DUrole{n}{X}}, \emph{\DUrole{n}{c}\DUrole{o}{=}\DUrole{default_value}{0}}, \emph{\DUrole{n}{p}\DUrole{o}{=}\DUrole{default_value}{2}}, \emph{\DUrole{n}{fitselect}\DUrole{o}{=}\DUrole{default_value}{\textquotesingle{}unrestricted\textquotesingle{}}}, \emph{\DUrole{n}{kernel}\DUrole{o}{=}\DUrole{default_value}{\textquotesingle{}triangular\textquotesingle{}}}, \emph{\DUrole{n}{vce}\DUrole{o}{=}\DUrole{default_value}{\textquotesingle{}jackknife\textquotesingle{}}}, \emph{\DUrole{n}{massPoints}\DUrole{o}{=}\DUrole{default_value}{True}}, \emph{\DUrole{n}{regularize}\DUrole{o}{=}\DUrole{default_value}{True}}, \emph{\DUrole{n}{nLocalMin}\DUrole{o}{=}\DUrole{default_value}{None}}, \emph{\DUrole{n}{nUniqueMin}\DUrole{o}{=}\DUrole{default_value}{None}}}{}
\pysigstopsignatures\begin{quote}\begin{description}
\item[{Parameters}] \leavevmode\begin{description}
\item[{\sphinxstylestrong{X: Numeric vector or one dimentional matrix/dataframe}}] \leavevmode
\sphinxAtStartPar
the running variable.

\item[{\sphinxstylestrong{c: Numeric}}] \leavevmode
\sphinxAtStartPar
specifies the threshold or cutoff value in the support of \sphinxstyleemphasis{X}. Default is \sphinxstyleemphasis{0}.

\item[{\sphinxstylestrong{p: Nonnegative integer,}}] \leavevmode
\sphinxAtStartPar
specifies the local polynomial order used to construct the density estimators. Default is \sphinxstyleemphasis{2} (local quadratic approximation).

\item[{\sphinxstylestrong{fitselect: String}}] \leavevmode
\sphinxAtStartPar
specifies the density estimation method. \sphinxstyleemphasis{unrestricted} (Default) for density estimation without any restrictions (two\sphinxhyphen{}sample, unrestricted inference). \sphinxstyleemphasis{restricted} for density estimation assuming equal distribution function and higher order dericatives.

\item[{\sphinxstylestrong{kernel: String}}] \leavevmode
\sphinxAtStartPar
specifies the kernel function used to construct the local polynomial estimators. Accepted kernels: \sphinxstyleemphasis{triangular} (Default), \sphinxstyleemphasis{epanechnikov} or \sphinxstyleemphasis{uniform}.

\item[{\sphinxstylestrong{vce: String}}] \leavevmode
\sphinxAtStartPar
specifies the procedure used to compute the variance\sphinxhyphen{}covariance matrix estimatior. \sphinxstyleemphasis{jackknife} (Default) for jackknife standard errors or \sphinxstyleemphasis{plugin} for asymptotic plug\sphinxhyphen{}in standard errors.

\item[{\sphinxstylestrong{massPoints: Boolean, Default *True*.}}] \leavevmode
\sphinxAtStartPar
Specifies wether to adjust for mass points in the data.

\item[{\sphinxstylestrong{regularize: Boolean, Default *True*.}}] \leavevmode
\sphinxAtStartPar
Specifies whether to conduct local sample size checking. When True, the bandwidth is chosen such that the local region includes at least \sphinxstyleemphasis{nLocalMin} observations and at least \sphinxstyleemphasis{nUniqueMin} unique observations.

\item[{\sphinxstylestrong{nLocalMin: Nonnegative integer}}] \leavevmode
\sphinxAtStartPar
specifies the minimum number of observations in each local neighbourhood. This option will be ignored if set to \sphinxstyleemphasis{0} or if \sphinxstyleemphasis{regularize=False}. Default is \sphinxstyleemphasis{20+p+1}.

\item[{\sphinxstylestrong{nUniqueMin: Nonnegative integer}}] \leavevmode
\sphinxAtStartPar
specifies the minimum number of unqieu observations in each local neighbourhood. This option will be ignored if set to \sphinxstyleemphasis{0} or if \sphinxstyleemphasis{regularize=False}. Default is \sphinxstyleemphasis{20+p+1}.

\end{description}

\item[{Returns}] \leavevmode\begin{description}
\item[{h}] \leavevmode
\sphinxAtStartPar
Bandwidths for density discontinuity test, left and right of the cutoff, asymptotic variance and bias.

\item[{n}] \leavevmode
\sphinxAtStartPar
full\sphinxhyphen{}full sample size, \sphinxstyleemphasis{left}/\sphinxstyleemphasis{right}: sample size to the left/right of the cutoff.

\item[{X\_min}] \leavevmode
\sphinxAtStartPar
Smallest observations to the left and right of the cutoff.

\item[{X\_max}] \leavevmode
\sphinxAtStartPar
Largest observations to the left and right of the cutoff.

\item[{options}] \leavevmode
\sphinxAtStartPar
other options passed to the function are also stored within the object.

\end{description}

\end{description}\end{quote}


\sphinxstrong{See also:}
\nopagebreak

\begin{description}
\item[{{\hyperref[\detokenize{rddensity:module-rddensity.rddensity}]{\sphinxcrossref{\sphinxcode{\sphinxupquote{rddensity.rddensity}}}}}}] \leavevmode
\item[{{\hyperref[\detokenize{rdplotdensity:module-rddensity.rdplotdensity}]{\sphinxcrossref{\sphinxcode{\sphinxupquote{rddensity.rdplotdensity}}}}}}] \leavevmode
\end{description}



\end{fulllineitems}



\subsubsection{Example}
\label{\detokenize{rdbwdensity:example}}
\begin{sphinxVerbatim}[commandchars=\\\{\}]
\PYG{g+gp}{\PYGZgt{}\PYGZgt{}\PYGZgt{} }\PYG{k+kn}{import} \PYG{n+nn}{numpy} \PYG{k}{as} \PYG{n+nn}{np}
\PYG{g+gp}{\PYGZgt{}\PYGZgt{}\PYGZgt{} }\PYG{k+kn}{from} \PYG{n+nn}{rddensity} \PYG{k+kn}{import} \PYG{n}{rdbwdensity}
\PYG{g+gp}{\PYGZgt{}\PYGZgt{}\PYGZgt{} }\PYG{n}{data} \PYG{o}{=} \PYG{n}{np}\PYG{o}{.}\PYG{n}{random}\PYG{o}{.}\PYG{n}{normal}\PYG{p}{(}\PYG{o}{\PYGZhy{}}\PYG{l+m+mf}{0.5}\PYG{p}{,}\PYG{l+m+mi}{1}\PYG{p}{,}\PYG{l+m+mi}{2000}\PYG{p}{)}
\PYG{g+gp}{\PYGZgt{}\PYGZgt{}\PYGZgt{} }\PYG{n}{est} \PYG{o}{=} \PYG{n}{rdbwdensity}\PYG{p}{(}\PYG{n}{data}\PYG{o}{=}\PYG{n}{data}\PYG{p}{,} \PYG{n}{vce}\PYG{o}{=}\PYG{l+s+s2}{\PYGZdq{}}\PYG{l+s+s2}{jackknife}\PYG{l+s+s2}{\PYGZdq{}}\PYG{p}{)}
\PYG{g+gp}{\PYGZgt{}\PYGZgt{}\PYGZgt{} }\PYG{n+nb}{print}\PYG{p}{(}\PYG{n+nb}{repr}\PYG{p}{(}\PYG{n}{est}\PYG{p}{)}\PYG{p}{)}
\end{sphinxVerbatim}

\sphinxstepscope


\subsection{rdplotdensity}
\label{\detokenize{rdplotdensity:rdplotdensity}}\label{\detokenize{rdplotdensity::doc}}
\sphinxAtStartPar
Density plotting for mainpulation testing


\subsubsection{Description}
\label{\detokenize{rdplotdensity:description}}
\sphinxAtStartPar
\sphinxcode{\sphinxupquote{rdplotdensity}} constructs density plots.
It is based on the local polunomial density estimator proposed in Cattaneo, Jansson and Ma (2020, 2021a).

\sphinxAtStartPar
Companion commands: \sphinxcode{\sphinxupquote{rdbwdensity}} for bandwidth selection and \sphinxcode{\sphinxupquote{rddensity}} for estimation.


\subsubsection{Details}
\label{\detokenize{rdplotdensity:details}}
\sphinxAtStartPar
Bias correction is only used for the construction of confidence intervals/bands, but not for point
estimation. The point estimates, denoted by \sphinxstyleemphasis{f\_p}, are constructed using local polynomial estimates
of order \sphinxstyleemphasis{p}, while the centering of the confidence intervals/bands, denoted by \sphinxstyleemphasis{f\_q}, are constructed
using local polynomial estimates of order \sphinxstyleemphasis{q}. The confidence intervals/bands take the form:
\sphinxstyleemphasis{{[}f\_q \sphinxhyphen{} cv * SE(f\_q) , f\_q + cv * SE(f\_q){]}}, where \sphinxstyleemphasis{cv} denotes the appropriate critical value and \sphinxstyleemphasis{SE(f\_q)}
denotes an standard error estimate for the centering of the confidence interval/band. As a result,
the confidence intervals/bands may not be centered at the point estimates because they have been bias\sphinxhyphen{}corrected.
Setting \sphinxstyleemphasis{q} and \sphinxstyleemphasis{p} to be equal results on centered at the point estimate confidence intervals/bands,
but requires undersmoothing for valid inference (i.e., (I)MSE\sphinxhyphen{}optimal bandwdith for the density point estimator
cannot be used). Hence the bandwidth would need to be specified manually when \sphinxstyleemphasis{q=p}, and the
point estimates will not be (I)MSE optimal. See Cattaneo, Jansson and Ma (2020a, 2020b) for details, and also
Calonico, Cattaneo, and Farrell (2018, 2020) for robust bias correction methods.

\sphinxAtStartPar
Sometimes the density point estimates may lie outside of the confidence intervals/bands, which can happen
if the underlying distribution exhibits high curvature at some evaluation point(s). One possible solution
in this case is to increase the polynomial order \sphinxstyleemphasis{p} or to employ a smaller bandwidth.


\subsubsection{References}
\label{\detokenize{rdplotdensity:references}}
\sphinxAtStartPar
Calonico, S., M. D. Cattaneo, and M. H. Farrell. 2018.
\sphinxhref{https://nppackages.github.io/references/Calonico-Cattaneo-Farrell\_2018\_JASA.pdf}{On the Effect of Bias Estimation on Coverage Accuracy in Nonparametric Inference}
\sphinxstyleemphasis{Journal of the American Statistical Association}, 113(522): 767\sphinxhyphen{}779.

\sphinxAtStartPar
Calonico, S., M. D. Cattaneo, and M. H. Farrell. 2020.
\sphinxhref{https://nppackages.github.io/references/Calonico-Cattaneo-Farrell\_2020\_CEopt.pdf}{Coverage Error Optimal Confidence Intervals for Local Polynomial Regression}
. Working paper.

\sphinxAtStartPar
Cattaneo M. D., M. Jansson, and X. Ma. 2018.
\sphinxhref{https://rdpackages.github.io/references/Cattaneo-Jansson-Ma\_2018\_Stata.pdf}{Manipulation Testing based on Density Discontinuity.}
\sphinxstyleemphasis{Stata Journal} 18(1): 234\sphinxhyphen{}261.

\sphinxAtStartPar
Cattaneo, M. D., M. Jansson, and X. Ma. 2020.
\sphinxhref{https://nppackages.github.io/references/Cattaneo-Jansson-Ma\_2020\_JASA.pdf}{Simple Local Polynomial Density Estimators}.
\sphinxstyleemphasis{Journal of the American Statistical Association}, 115(531): 1449\sphinxhyphen{}1455.

\sphinxAtStartPar
Cattaneo, M. D., M. Jansson, and X. Ma. 2021a.
\sphinxhref{https://nppackages.github.io/references/Cattaneo-Jansson-Ma\_2021\_JoE.pdf}{Local Regression Distribution Estimators}
\sphinxstyleemphasis{Journal of Econometrics}, forthcoming.

\sphinxAtStartPar
Cattaneo M. D., M. Jansson, and X. Ma. 2022.
\sphinxhref{https://rdpackages.github.io/references/Cattaneo-Jansson-Ma\_2022\_JSS.pdf}{lpdensity: Local Polynomial Density Estimation and Inference.}
\sphinxstyleemphasis{Journal of Statistical Software} Forthcoming.


\subsubsection{Authors}
\label{\detokenize{rdplotdensity:authors}}
\sphinxAtStartPar
Matias D. Cattaneo, Princeton University. (\sphinxhref{mailto:cattaneo@princeton.edu}{cattaneo@princeton.edu}).

\sphinxAtStartPar
Rajita Chandak (maintainer), Princeton University. (\sphinxhref{mailto:rchandak@princeton.edu}{rchandak@princeton.edu}).

\sphinxAtStartPar
Michael Jansson, University of California Berkeley. (\sphinxhref{mailto:mjansson@econ.berkeley.edu}{mjansson@econ.berkeley.edu}).

\sphinxAtStartPar
Xinwei Ma (maintainer), University of California San Diego. (\sphinxhref{mailto:x1ma@ucsd.edu}{x1ma@ucsd.edu}).

\phantomsection\label{\detokenize{rdplotdensity:module-rddensity.rdplotdensity}}\index{module@\spxentry{module}!rddensity.rdplotdensity@\spxentry{rddensity.rdplotdensity}}\index{rddensity.rdplotdensity@\spxentry{rddensity.rdplotdensity}!module@\spxentry{module}}\index{rdplotdensity() (in module rddensity.rdplotdensity)@\spxentry{rdplotdensity()}\spxextra{in module rddensity.rdplotdensity}}

\begin{fulllineitems}
\phantomsection\label{\detokenize{rdplotdensity:rddensity.rdplotdensity.rdplotdensity}}
\pysigstartsignatures
\pysiglinewithargsret{\sphinxcode{\sphinxupquote{rddensity.rdplotdensity.}}\sphinxbfcode{\sphinxupquote{rdplotdensity}}}{\emph{\DUrole{n}{rdd}}, \emph{\DUrole{n}{X}}, \emph{\DUrole{n}{plotRange}\DUrole{o}{=}\DUrole{default_value}{None}}, \emph{\DUrole{n}{plotN}\DUrole{o}{=}\DUrole{default_value}{{[}10{]}}}, \emph{\DUrole{n}{plotGrid}\DUrole{o}{=}\DUrole{default_value}{{[}\textquotesingle{}es\textquotesingle{}, \textquotesingle{}qs\textquotesingle{}{]}}}, \emph{\DUrole{n}{alpha}\DUrole{o}{=}\DUrole{default_value}{0.05}}, \emph{\DUrole{n}{plottype}\DUrole{o}{=}\DUrole{default_value}{\textquotesingle{}line\textquotesingle{}}}, \emph{\DUrole{n}{CItype}\DUrole{o}{=}\DUrole{default_value}{\textquotesingle{}region\textquotesingle{}}}, \emph{\DUrole{n}{CIuniform}\DUrole{o}{=}\DUrole{default_value}{False}}, \emph{\DUrole{n}{CIsimul}\DUrole{o}{=}\DUrole{default_value}{2000}}, \emph{\DUrole{n}{CIshade}\DUrole{o}{=}\DUrole{default_value}{0.2}}, \emph{\DUrole{n}{bwselect}\DUrole{o}{=}\DUrole{default_value}{None}}, \emph{\DUrole{n}{hist}\DUrole{o}{=}\DUrole{default_value}{True}}, \emph{\DUrole{n}{histBreaks}\DUrole{o}{=}\DUrole{default_value}{None}}, \emph{\DUrole{n}{histfillshade}\DUrole{o}{=}\DUrole{default_value}{0.1}}, \emph{\DUrole{n}{histlinecol}\DUrole{o}{=}\DUrole{default_value}{\textquotesingle{}white\textquotesingle{}}}, \emph{\DUrole{n}{title}\DUrole{o}{=}\DUrole{default_value}{None}}, \emph{\DUrole{n}{xlabel}\DUrole{o}{=}\DUrole{default_value}{None}}, \emph{\DUrole{n}{ylabel}\DUrole{o}{=}\DUrole{default_value}{None}}, \emph{\DUrole{n}{legendtitle}\DUrole{o}{=}\DUrole{default_value}{None}}, \emph{\DUrole{n}{legendgroups}\DUrole{o}{=}\DUrole{default_value}{None}}}{}
\pysigstopsignatures\begin{quote}\begin{description}
\item[{Parameters}] \leavevmode\begin{description}
\item[{\sphinxstylestrong{rdd}}] \leavevmode{[}\sphinxtitleref{rddensity} object{]}
\sphinxAtStartPar
returned by \sphinxstyleemphasis{rddensity}.

\item[{\sphinxstylestrong{X: Numeric vector or one dimentional matrix/dataframe}}] \leavevmode
\sphinxAtStartPar
the running variable.

\item[{\sphinxstylestrong{plotRange: Numeric.}}] \leavevmode
\sphinxAtStartPar
Specifies the lower and upper bound of the plotting region. Default is \sphinxstyleemphasis{{[}c\sphinxhyphen{}3hl, c+3hr{]}}

\item[{\sphinxstylestrong{plotN: Numeric.}}] \leavevmode
\sphinxAtStartPar
Specifies the bumber of grid points used for plotting on the two sides of the cutoff. Default is 10 on each side.

\item[{\sphinxstylestrong{plotGrid: String.}}] \leavevmode
\sphinxAtStartPar
Specifies position of grid points. \sphinxstyleemphasis{es} for evenly spaced, \sphinxstyleemphasis{qs} for quantile spaced.

\item[{\sphinxstylestrong{alpha: Numeric scalar between 0 and 1.}}] \leavevmode
\sphinxAtStartPar
The significance level for plotting confidence regions.

\item[{\sphinxstylestrong{plottype: String.}}] \leavevmode
\sphinxAtStartPar
\sphinxstyleemphasis{“line”}, \sphinxstyleemphasis{“points”} or \sphinxstyleemphasis{“both”} specifies how the estiamtes are plotted.

\item[{\sphinxstylestrong{CItype: String.}}] \leavevmode
\sphinxAtStartPar
\sphinxstyleemphasis{“region”} (default),  \sphinxstyleemphasis{“line”} or \sphinxstyleemphasis{“ebar”}, how the confidence region will be plotted.

\item[{\sphinxstylestrong{CIuniform: Boolean (default False)}}] \leavevmode
\sphinxAtStartPar
plotting pointwise confidence intervals or uniform confidence bands.

\item[{\sphinxstylestrong{CIsimul: Positive integer.}}] \leavevmode
\sphinxAtStartPar
Number of simulations used to construct confidence intervals (default 2000). Ignored if \sphinxstyleemphasis{CIuniform} is False.

\item[{\sphinxstylestrong{CIshade: Numeric, between 0 and 1.}}] \leavevmode
\sphinxAtStartPar
Opaquness of confidence region. Default is \sphinxstyleemphasis{0.2}.

\item[{\sphinxstylestrong{bwselect: String.}}] \leavevmode
\sphinxAtStartPar
Method for data\sphinxhyphen{}driven bandwidth selection. Default uses bandwidth from \sphinxstyleemphasis{rdd}. \sphinxstyleemphasis{“mse\sphinxhyphen{}dpi”}\sphinxhyphen{} mean squared error\sphinxhyphen{}optimal bandwidth selected for each grid points, \sphinxstyleemphasis{“imse\sphinxhyphen{}dpi”}\sphinxhyphen{} integrated MSE\sphinxhyphen{}optimal bandiwdth, common for all grid points, \sphinxstyleemphasis{“mse\sphinxhyphen{}rot”}\sphinxhyphen{} rule\sphinxhyphen{}of\sphinxhyphen{}thumb bandiwdth with Gaussian reference model, \sphinxstyleemphasis{“imse\sphinxhyphen{}rot”}\sphinxhyphen{}integrated rule\sphinxhyphen{}of\sphinxhyphen{}thumb bandiwdht with Gaussian reference model.

\item[{\sphinxstylestrong{hist: Boolean (default True).}}] \leavevmode
\sphinxAtStartPar
Adds histgram in background of plot.

\item[{\sphinxstylestrong{histBreaks: Numeric vector.}}] \leavevmode
\sphinxAtStartPar
breakpoints between histogram bars.

\item[{\sphinxstylestrong{histfillshade: Numeric between 0 and 1.}}] \leavevmode
\sphinxAtStartPar
Opaqueness of histrogram. Default is \sphinxstyleemphasis{0.1}.

\item[{\sphinxstylestrong{title: String.}}] \leavevmode
\sphinxAtStartPar
Title of the plot

\item[{\sphinxstylestrong{xlabel: String.}}] \leavevmode
\sphinxAtStartPar
Label for x\sphinxhyphen{}axis.

\item[{\sphinxstylestrong{ylabel: String.}}] \leavevmode
\sphinxAtStartPar
Label for y\sphinxhyphen{}axis.

\item[{\sphinxstylestrong{legendTitle: String.}}] \leavevmode
\sphinxAtStartPar
Title of legend.

\end{description}

\item[{Returns}] \leavevmode\begin{description}
\item[{plot: plotnine object.}] \leavevmode
\sphinxAtStartPar
Can be customized further with plotnine options.

\end{description}

\end{description}\end{quote}


\sphinxstrong{See also:}
\nopagebreak

\begin{description}
\item[{{\hyperref[\detokenize{rddensity:module-rddensity.rddensity}]{\sphinxcrossref{\sphinxcode{\sphinxupquote{rddensity.rddensity}}}}}}] \leavevmode
\item[{{\hyperref[\detokenize{rdbwdensity:module-rddensity.rdbwdensity}]{\sphinxcrossref{\sphinxcode{\sphinxupquote{rddensity.rdbwdensity}}}}}}] \leavevmode
\end{description}



\end{fulllineitems}



\subsubsection{Example}
\label{\detokenize{rdplotdensity:example}}
\begin{sphinxVerbatim}[commandchars=\\\{\}]
\PYG{g+gp}{\PYGZgt{}\PYGZgt{}\PYGZgt{} }\PYG{k+kn}{import} \PYG{n+nn}{numpy} \PYG{k}{as} \PYG{n+nn}{np}
\PYG{g+gp}{\PYGZgt{}\PYGZgt{}\PYGZgt{} }\PYG{k+kn}{from} \PYG{n+nn}{rddensity} \PYG{k+kn}{import} \PYG{n}{rddensity}
\PYG{g+gp}{\PYGZgt{}\PYGZgt{}\PYGZgt{} }\PYG{k+kn}{from} \PYG{n+nn}{rddensity} \PYG{k+kn}{import} \PYG{n}{rdplotdensity}
\PYG{g+gp}{\PYGZgt{}\PYGZgt{}\PYGZgt{} }\PYG{n}{data} \PYG{o}{=} \PYG{n}{np}\PYG{o}{.}\PYG{n}{random}\PYG{o}{.}\PYG{n}{normal}\PYG{p}{(}\PYG{o}{\PYGZhy{}}\PYG{l+m+mf}{0.5}\PYG{p}{,}\PYG{l+m+mi}{1}\PYG{p}{,}\PYG{l+m+mi}{2000}\PYG{p}{)}
\PYG{g+gp}{\PYGZgt{}\PYGZgt{}\PYGZgt{} }\PYG{n}{rdd} \PYG{o}{=} \PYG{n}{rddensity}\PYG{p}{(}\PYG{n}{X}\PYG{o}{=}\PYG{n}{data}\PYG{p}{,} \PYG{n}{vce}\PYG{o}{=}\PYG{l+s+s2}{\PYGZdq{}}\PYG{l+s+s2}{jackknife}\PYG{l+s+s2}{\PYGZdq{}}\PYG{p}{)}
\PYG{g+gp}{\PYGZgt{}\PYGZgt{}\PYGZgt{} }\PYG{n}{plot1} \PYG{o}{=} \PYG{n}{rdplotdensity}\PYG{p}{(}\PYG{n}{rdd}\PYG{p}{,} \PYG{n}{data}\PYG{p}{)}
\end{sphinxVerbatim}


\renewcommand{\indexname}{Python Module Index}
\begin{sphinxtheindex}
\let\bigletter\sphinxstyleindexlettergroup
\bigletter{r}
\item\relax\sphinxstyleindexentry{rddensity.rdbwdensity}\sphinxstyleindexpageref{rdbwdensity:\detokenize{module-rddensity.rdbwdensity}}
\item\relax\sphinxstyleindexentry{rddensity.rddensity}\sphinxstyleindexpageref{rddensity:\detokenize{module-rddensity.rddensity}}
\item\relax\sphinxstyleindexentry{rddensity.rdplotdensity}\sphinxstyleindexpageref{rdplotdensity:\detokenize{module-rddensity.rdplotdensity}}
\end{sphinxtheindex}

\renewcommand{\indexname}{Index}
\printindex
\end{document}